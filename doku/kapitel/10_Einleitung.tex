%!TEX root=../ecdsa.tex
% \lipsum
\chapter{Einleitung}
Mit wachsender Digitalisierung wächst auch die Sorge, was mit den eigenen Daten passiert und ob diese nach der Übertragung über ein Medium nicht durch andere manipuliert worden sind. Um die Sicherheit bzw. Authentizität der Daten zu gewährleisten, werden die Daten mit Hilfe eines Passwortes, dem Schlüssel, signiert, sodass ein Empfänger, der ebenfalls den Schlüssel kennt, die Daten verifizieren kann. Der sich dahinter verbergende Algorithmus wird Digitaler Signatur Algorithmus (engl. Digital Signature Algorithm, DSA) genannt.
\\ \\
Will ein Angreifer trotz Signatur eine Nachricht manipulieren, kann dieser über eine Brute-Force-Attacke versuchen, den Schlüssel durch ein Testen jeder erdenklichen Kombination zu erraten. Gelingt dies, kann nach der Manipulation eine neue gültige Signatur erstellt werden. Um diesem Angriffsszenario entgegen zu wirken, werden lange Schlüssel verwendet, sodass eine solche Attacke nicht in akzeptabler Zeit zu bewerkstelligen ist. Dies hat jedoch zur Folge, dass mit steigender Schlüssellänge auch die Zeit steigt, die für das Erstellen der Signatar bzw. die Verifizierung benötigt wird. 
\\ \\
Eine mögliche Lösung zur Reduzierung der Schlüssellänge und damit der Laufzeit besteht im Einsatz optimierter Verfahren wie die elliptischen Kurven, die im Gegensatz zu Verfahren wie RSA (engl. Random Sequential Adsorption), deutlich geringe Schlüssellängen benötigen. Diese Arbeit widmet sich den zuvor erwähnten elliptischen Kurven mit dem Ziel, eine effiziente Hardware-Implementierung zu entwickeln, die eine klassische Software-Variante übertrifft.
