\chapter{Zusammenfassung \& Ausblick} 
\label{sec:fazit}
In dieser Arbeit wurden die Grundlagen der Elliptischen-Kurven-Kryptographie erläutert und eine Implementierung des ECDSA-Algorithmus in VHDL mit dem Ziel erstellt, die Geschwindgkeit mit der einer C-Implementierung zu vergleichen.
\\ \\
Die Ergebnisse aus Kapitel \ref{sec:messung-results} zeigen, dass eine Implementierung in VHDL eine Implementierung in C in puncto Geschwindigkeit deutlich verbessern kann und das bereits bei einer geringen Schlüssellänge und rudimentären Optimierungen. Zu berücksichtigen gilt, dass die C-Implementierung von optimalen Ergebnissen weit entfernt ist und weiter optimiert werden kann. So verwendet die C-Variante eine für Hardware optimierte Implementierung in Form von Bitfeldern und keine numerische Darstellung.
\\ \\
Trotz der positiven Ergebnisse wurde leider bei der Validierung der Ergebnisse festgestellt, dass die verwendete C-Implementierung den ECDSA-Algorithmus fehlerhaft implementiert hat. Auch nach verschiedenen Testläufen, studieren verschiedener Paper und Berechnungen von Hand konnte die Ursache des Problems innerhalb der zur Verfügung stehenden Zeit nicht identifiziert werden. Die Vermutung liegt Nahe, dass entweder die Kurvenparameter selbst oder eine fehlerhafte Berücksichtigung von Randbedingungen den Fehler verursacht. Ein allgemeiner Fehler durch falsche Implementierung der Abläufe ist auszuschließen. Trotz dieser Problematik wurden zwischen beiden Implementierungen identische Werte berechnet, sodass dennoch ein Vergleich der Laufzeit möglich ist.


