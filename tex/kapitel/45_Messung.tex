%!TEX root = ../ecdsa.tex

\chapter{Messung \& Vergleich} \label{sec:messung}

%%%%%%%%%%%%%%%%%%%%%%%%%%%%%%%%%%%%%%%%%%%%%%%%%%%%%%%%%%%%%
\section{FPGA-Implementierung}
Zur Messungen der Laufzeit werden Daten über die UART-Schnittstelle gesendet und die Zeit gemessen, bis das Ergebnis empfangen wurde. Da das Signieren bzw. Verifizieren, wie in Abschnitt \ref{sec:uartimpl} beschrieben, durch separate Kommandos möglich ist, kann so die Laufzeit pro Funktion ermittelt werden. 
\\ \\
Die Kommunikation mit der FPGA-Implementierung wird über ein selbstgeschriebenes Python-Skript abgewickelt, welches einen zufälligen 163-Bit Hash erzeugt und via UART versendet. Nach dem Empfangen der erzeugten Signatur wird diese erneut zum FPGA geschickt und verifiziert. Dieser Ablauf wird für 1000 Messungen wiederholt und anschließend der Mittelwert gebildet. Die Ergebnisse sind in Tabelle \ref{vhdl-messung} zu finden.
\\ \\
\begin{table}[h]
	\centering 
	\begin{tabular}{ | p{3cm} | p{6cm} | }
		\hline
		\textbf{Funktion} & \textbf{Zeit} \\
		\hline
		Signieren & 77,0 ms \\
		\hline
		Verifizieren & 130,1 ms \\
		\hline
	\end{tabular}
	\caption{Messergebnisse der VHDL-Implementierung}
	\label{vhdl-messung}
\end{table}

Bei den Messergebnissen ist zu beachten, dass diese die Zeit zur Übertragung der Daten beinhalten, die je nach verwendeter Funktion (Signieren oder Verifizieren) stark variieren können. Für eine exakte Bestimmung der Zeiten bedarf es noch einer Bereinigung der Messergebnisse. Dazu wird rechnerisch ermittelt, wie viel Zeit für die Übertragung benötigt wird und dieser Wert von den gemessenen Zeiten abgezogen. 
\\ \\
Die serielle Übertragung findet mit einer Geschwindigkeit von 9600 Baud statt. Ein Symbol liegt dadurch für die Signaldauer von $1/9600$ Sekunden, also etwa 104$\mu$s, auf der Leitung an. Zusätzlich zu den 8 Daten-Bits werden pro Übertragung noch das Start-Bit und das Stopp-Bit übermittelt. Somit werden Für die 10 Symbole ca. 1,04 ms benötigt. Da die internen Verzögerungen zwischen der UART-Schnittstelle und der ECDSA-Implementierung zu vernachlässigen sind, wird lediglich die UART-Kommunkation berücksichtigt. Dies führt zu den folgenden Messergebnissen:
\\
\begin{itemize}
	\item \textbf{Signieren:}\\
\textit{Senden:} 
1 Byte Modus, 21 Byte Message\\
$\Rightarrow$ 22 Byte * 10 Symbole * 104$\mu$s = 22,92ms\\
\textit{Empfangen:} 
2x21 Byte Punkte der ECC-Funktion\\
$\Rightarrow$ 42 Byte  * 10 Symbole * 104$\mu$s = 43,75ms \\
%\textbf{Nettozeit Signieren} = 123,0ms - 22,92ms - 43,75ms = \textbf{56,3ms}
\textbf{Nettozeit Signieren} = 77,0ms - 22,92ms - 43,75ms = \textbf{10,3ms}
	
	\item \textbf{Verifizieren:}\\
\textit{Senden:}\\
1 Byte Modus, 2x21 Byte ECC-Punkte, 21 Byte Message\\
$\Rightarrow$ 64 Byte * 10 Symbole * 104 $\mu$s = 66,67 ms\\
\textit{Empfangen:}\\
1 Byte für True/False
$\Rightarrow$ 1 Byte  * 10 Symbole * 104 $\mu$s = 1,04 ms \\
%\textbf{Nettozeit Verifizieren} = 74 ms - 66,67 ms - 1,04 ms = \textbf{6,3 ms}\\
\textbf{Nettozeit Verifizieren} = 130,1 ms - 66,67 ms - 1,04 ms = \textbf{62,4 ms}\\
\end{itemize}

Die oben angegebenen errechneten Werte weichen höchstwahrscheinlich stark von den \textit{tatsächlichen} Laufzeiten des Algorithmus auf dem Board ab. Grund hierfür sind die verschiedenen Puffer-Mechanismen der verwendeten ``PySerial''-Bibliothek für serielle Schnittstellen in Python sowie an der Pufferung des Betriebssystems. 
\\ \\
Um exakte Laufzeiten der beiden Modi zu ermitteln, wird das Signal \texttt{ready\_o} (vgl. Kap. \ref{vhdl-impl-general}), welches die Aktivität des Kernmoduls beschreibt, über einen Pin nach außen geführt und mit einem Oszilloskop gemessen. Diese Messwerte liegen deutlich näher aneinander und liegen zudem um Faktor 4 bis 24 unter den berechneten Nettoergebnissen der Python-Messung. \\

\begin{table}[h]
	\centering 
	\begin{tabular}{ | p{3cm} | p{3cm} | p{3cm} | }
		\hline
		\textbf{Funktion} & \textbf{Python} & \textbf{Oszilloskop} \\
		\hline
		Signieren & $\sim$ 10,3 ms & 2,366 ms\\
		\hline
		Verifizieren & $\sim$ 62,4 ms & 2,436 ms\\
		\hline
	\end{tabular}
	\caption{Messergebnisse der VHDL-Implementierung mit einem Oszilloskop}
	\label{vhdl-messung-2}
\end{table}

%%%%%%%%%%%%%%%%%%%%%%%%%%%%%%%%%%%%%%%%%%%%%%%%%%%%%%%%%%%%%
\section{C-Implementierung}
Für die Messung des C-Codes wurde die vorhandene C-Implementierung um Code zum Messen der Funktionen ergänzt und auf einem Rechner mit dem Betriebssystem Linux kompiliert und gestartet. Eine Übersicht der verwendeten Hardware ist in Tabelle \ref{c-impl-hardware} zu finden. \\

\begin{table}[h]
	\centering 
	\begin{tabular}{ | p{3cm} | p{6cm} | }
		\hline
		\textbf{Funktion} & \textbf{Zeit} \\
		\hline
		Signieren & 169.1 ms \\
		\hline
		Verifizieren & 340.5 ms \\
		\hline
	\end{tabular}
	\caption{Messergebnisse der C-Implementierung}
	\label{c-messung}
\end{table}

Wie bereits bei der VHDL-Implementierung wurden wieder $1000$ Messungen durchgeführt, bei denen im Mittel die Ergebnisse aus Tabelle \ref{c-messung} erzielt wurden. \\

\begin{table}
	\centering 
	\begin{tabular}{ | l | l | }
		\hline
		Bezeichnung & Beschreibung \\
		\hline
		CPU & Intel i5-5200U DualCore (2.20 bis 2.70 GHz, 3MB Cache) \\ 
		RAM & 8GB DDR3L-1600 \\
		HDD & 256GB SSD  \\
		Grafik & Intel® HD 5500 Grafik \\
		OS & Linux \\
		\hline
	\end{tabular}
	\caption{Konfiguration der verwendeten Hardware}
	\label{c-impl-hardware}
\end{table}


%%%%%%%%%%%%%%%%%%%%%%%%%%%%%%%%%%%%%%%%%%%%%%%%%%%%%%%%%%%%%
\section{Gegenüberstellung \& Bewertung}
\label{sec:messung-results}

Bei den Messergebnissen aus Tabelle \ref{messung-results} ist klar zu erkennen, dass die VHDL-Implementierung je nach Funktion eine Performanz-Steigerung zwischen 98,6 \% - 99,3 \% aufweist. \\

\begin{table} [h]
	\centering 
	\begin{tabular}{ | p{3cm} | p{2cm} | p{2cm} | p{2cm} | }
		\hline
		 & \textbf{VHDL} & \textbf{C} & \textbf{Gewinn} \\
		\hline
		\textbf{Signieren} & 2,366 ms & 169,1 ms & 98,6 \% \\
		\hline
		\textbf{Verifizieren} &  2,436 ms & 340,5 ms & 99,3 \% \\
		\hline
	\end{tabular}
	\caption{Gegenüberstelung der Messergebnisse von der C- und VHDL-Implementierung}
	\label{messung-results}
\end{table}

Bei den Messungen gilt zu berücksichtigen, dass der Performanz-Gewinn bereits bei einer geringen Schlüssellänge von 163 Bit erzielt wurde. Da bei einem Erhöhen der Schlüssellänge die Vorteile der Hardware-Implementierung mehr zum Tragen kommen, ist mit weiteren Leistungssteigerungen zu rechnen. Außerdem wurde, wie in Kapitel \ref{sec:impl} beschrieben, lediglich eine triviale VHDL-Implementierung umgesetzt. Durch ein Austauschen der generischen Implementierung durch eine statische optimierte Verdrahtung sollten sich weitere Performanz-Verbesserungen erzielen lassen. Weitere Optimierungen können eine Anpassung der Algorithmen (z.B. Montgomery Multiplizierer) sein.
\\ \\
Abschließend sei erwähnt, dass die C-Implementierung \cite{kewish} eine auf Hardware optimierte Implementierung (Bit-Felder) verwendet. Es ist daher davon auszugehen, dass sich auch die C-Ergebnisse verbessern lassen, indem beispielsweise auf numerische Verfahren zurückgegriffen wird. \\

%\begin{figure}[H]
%	\centering
%  \includegraphics[width=1.00\textwidth]{bilder/filename}
%	\caption{caption}
%	\label{label}
%\end{figure}
